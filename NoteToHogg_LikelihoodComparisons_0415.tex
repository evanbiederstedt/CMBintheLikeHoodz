\documentclass[a4paper, 11pt]{article}
\usepackage{comment} % enables the use of multi-line comments (\ifx \fi) 
\usepackage{lipsum} %This package just generates Lorem Ipsum filler text. 
\usepackage{fullpage} % changes the margin
\usepackage{amsfonts} %allows use of mathbb
\usepackage{gensymb} %allows use of degree symbol, otherwise use ^{\circ}
\usepackage{amsmath} %allows use of \text{} command

\begin{document}
%Header-Make sure you update this information!!!!
%\noindent
%\large\textbf{Likelihood} \hfill \textbf{Evan Biederstedt} \\
%\normalsize Discuss First paper, likelihood, etc. 
%\hfill \ 12/05/2014 \\

\section*{E-mail note to Hogg, Likelihood comparisons, 4/15}

The code is a simple procedural program. We just want to expliticitly test whether we are doing the correct method to create the dot product matrix. 

The first covariance matrix is defined by taking an array of temperature values, $T$, substracting the mean, and multiplying with itself. The formalism is taken directly from arXiv:1409.7718, 2015, "Comparing Planck and WMAP: Maps, Spectra, and Parameters", D. Larson et al. Appendix B, (B2)

$$
C_{ij} = <\Delta T_i \Delta T_j > = \frac{1}{N_{\text{pix}}} \sum^{\text{pix}}_{p=1} (T^i(p)-\bar{T}^i) (T^j(p)-\bar{T}^j)
$$






The second covariance matrix is defined as 

$$
C_{ij}=\sum^{l}_{l=0} \frac{2l+1}{4/pi} C^{\text{theor}}_{l}P_{l}(\cos\alpha_{ij})
$$

where $C_{ij}$ is the covariance between pixel $i$ and pixel $j$,  $C^{\text{theor}}_{l}$ is the theoretical value of $C_l$, $P_{l}$  are the Legendre polynomial, and $\alpha_{ij}$ is the angle between. The dot product is defined by unit vectors, $\cos\alpha_{ij}=\hat{n}_{i}\cdot\hat{n}_{j}$. 




\end{document} 